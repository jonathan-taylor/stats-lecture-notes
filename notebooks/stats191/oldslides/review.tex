   \documentclass[handout]{beamer}



   \mode<presentation>
   {
     \usetheme{PaloAlto}
   \setbeamertemplate{footline}[page number]

     \setbeamercolor{sidebar}{bg=white, fg=black}
     \setbeamercolor{frametitle}{bg=white, fg=black}
     % or ...
     \setbeamercolor{logo}{bg=white}
     \setbeamercolor{block body}{parent=normal text,bg=white}
     \setbeamercolor{author in sidebar}{fg=black}
     \setbeamercolor{title in sidebar}{fg=black}


     \setbeamercolor*{block title}{use=structure,fg=structure.fg,bg=structure.fg!20!bg}
     \setbeamercolor*{block title alerted}{use=alerted text,fg=alerted text.fg,bg=alerted text.fg!20!bg}
     \setbeamercolor*{block title example}{use=example text,fg=example text.fg,bg=example text.fg!20!bg}


     \setbeamercolor{block body}{parent=normal text,use=block title,bg=block title.bg!50!bg}
     \setbeamercolor{block body alerted}{parent=normal text,use=block title alerted,bg=block title alerted.bg!50!bg}
     \setbeamercolor{block body example}{parent=normal text,use=block title example,bg=block title example.bg!50!bg}

     % or ...

     \setbeamercovered{transparent}
     % or whatever (possibly just delete it)
     \logo{\resizebox{!}{1.5cm}{\href{\basename{R}}{\includegraphics{image}}}}
   }

   \mode<handout>
   {
     \usetheme{PaloAlto}
     \usecolortheme{default}
     \setbeamercolor{sidebar}{bg=white, fg=black}
     \setbeamercolor{frametitle}{bg=white, fg=black}
     % or ...
     \setbeamercolor{logo}{bg=white}
     \setbeamercolor{block body}{parent=normal text,bg=white}
     \setbeamercolor{author in sidebar}{fg=black}
     \setbeamercolor{title in sidebar}{fg=black}
     \setbeamercovered{transparent}
     % or whatever (possibly just delete it)
     \logo{}
   }

   \usepackage{epsdice,listings,epic}
   \usepackage[latin1]{inputenc}
   \usepackage{graphicx}
   \usepackage{amsmath,eepic,epic,algorithm}

   \newcommand{\figslide}[3]{
   \begin{frame}
   \frametitle{#1}
     \begin{center}
     \resizebox{!}{2.7in}{\includegraphics{#2}}    
     \end{center}
   {#3}
   \end{frame}
   }

   \newcommand{\fighslide}[4]{
   \begin{frame}
   \frametitle{#1}
     \begin{center}
     \resizebox{!}{#4}{\includegraphics{#2}}    
     \end{center}
   {#3}
   \end{frame}
   }

   \newcommand{\figwref}[1]{
   \href{#1}{\tiny \tt #1}}

   \newcommand{\unsupervised}[1]{{\color{red} #1}}
   \newcommand{\supervised}[1]{{\color{green} #1}}
   \newcommand{\argmax}{\mathop{\mathrm{argmax}}}
   \newcommand{\argmin}{\mathop{\mathrm{argmin}}}
   \newcommand{\minimize}{\mathop{\mathrm{minimize}}}
   \newcommand{\maximize}{\mathop{\mathrm{maximize}}}

   \newcommand{\B}[1]{\beta_{#1}}
   \newcommand{\Bh}[1]{\widehat{\beta}_{#1}}
   \newcommand{\V}{\text{Var}}
   \newcommand{\Cov}{\text{Cov}}
   \newcommand{\Vh}{\widehat{\V}}
   \newcommand{\s}{\sigma}
   \newcommand{\sh}{\widehat{\sigma}}

   \newcommand{\argmax}[1]{\mathop{\text{argmax}}_{#1}}
   \newcommand{\argmin}[1]{\mathop{\text{argmin}}_{#1}}
   \newcommand{\Ee}{\mathbb{E}}
   \newcommand{\Pp}{\mathbb{P}}
   \newcommand{\real}{\mathbb{R}}
   \newcommand{\Ybar}{\overline{Y}}
   \newcommand{\Yh}{\widehat{Y}}
   \newcommand{\Xbar}{\overline{X}}
   \newcommand{\Tr}{\text{Tr}}


   \newcommand{\model}{{\cal M}}

   \newcommand{\figvskip}{-0.7in}
   \newcommand{\fighskip}{-0.3in}
   \newcommand{\figheight}{3.5in}

   \newcommand{\Rcode}[1]{{\bf \tt #1 }}
   \newcommand{\Rtcode}[1]{{\tiny \bf \tt #1 }}
   \newcommand{\Rscode}[1]{{\small \bf \tt #1 }}

   \newcommand{\RR}{{\tt R} \;}
   \newcommand{\basename}[1]{http://stats191.stanford.edu/#1}
   \newcommand{\dataname}[1]{\basename{data/#1}}
   \newcommand{\Rname}[1]{\basename{R/#1}}

   \newcommand{\mycolor}[1]{{\color{blue} #1}}
   \newcommand{\basehref}[2]{\href{\basename{#1}}{\mycolor{#2}}}
   \newcommand{\Rhref}[2]{\href{\basename{R/#1}}{\mycolor{#2}}}
   \newcommand{\datahref}[2]{\href{\dataname{#1}}{\mycolor{#2}}}
   \newcommand{\X}{\pmb{X}}
   \newcommand{\Y}{\pmb{Y}}
   \newcommand{\be}{\pmb{varepsilon}}
   \newcommand{\logit}{\text{logit}}


   \title{Statistics 191: Introduction to Applied Statistics}
   \subtitle{Review}
   \author{\copyright Jonathan Taylor \\
   }
   %}


   \begin{document}

   \begin{frame}
   \titlepage
   \end{frame}

   %%%%%%%%%%%%%%%%%%%%%%%%%%%%%%%%%%%%%%%%%%%%%%%%%%%%%%%%%%%%

   \begin{frame} 

   \begin{block}
   {Outline}
   \begin{itemize}

   \item What is a regression model?
   \item Descriptive statistics -- numerical
   \item Descriptive statistics -- graphical
   \item Inference about a population mean
   \item Difference between two population means
   \end{itemize}
   \end{block}
   \end{frame}

   %%%%%%%%%%%%%%%%%%%%%%%%%%%%%%%%%%%%%%%%%%%%%%%%%%%%%%%%%%%%

   \begin{frame} 

   \begin{block}
   {What is course about?}
   \begin{itemize}
   \item It is a course on applied statistics.

   \item Hands-on: we use \href{http://cran.r-project.org}{R}, an open-source statistics software environment.

   \item We will start out with a review of introductory statistics to see {\tt R} in action.
   \item Main topic is ``(linear) regression models'': these are the {\em bread and butter} of applied statistics.

   \end{itemize}
   \end{block}

   \begin{block}
   {What is a ``regression'' model? }
   A regression model is a model of the relationships between some
   {\em covariates (predictors)} and an {\em outcome}.
   Specifically, regression is a model of the {\em average} outcome {\em given}
   the covariates.
   \end{block}
   \end{frame}

   %%%%%%%%%%%%%%%%%%%%%%%%%%%%%%%%%%%%%%%%%%%%%%%%%%%%%%%%%%%%

   \begin{frame} 

   \begin{block}{Heights of couples}
   \begin{itemize}

   \item To study height of the wife in a couple, based on the
   husband's height and her parents height: {\tt Wife} is the
   outcome, and the covariates are {\tt Husband, Mother, Father}.


   \item A mathematical  model, using only {\tt Husband}'s height:
   $$
   {\tt Wife} = f({\tt Husband}) + \varepsilon$$
   where $f$ gives the average height of the wife
   of a man of height {\tt Husband} and
   $\varepsilon$ is ``error'': not {\em every} man of height of
   {\tt Husband} marries
   a woman of height $f({\tt Husband})$.

   \item A statistical question: is there {\em any}
   relationship between covariates and outcomes -- is $f$ just a constant?

   \item Here is some  \href{http://stats191.stanford.edu/review.html}{data}
   using only {\tt Husband}'s height.
   \end{itemize}

   \end{block}
   \end{frame}

   %CODE
       % # Load and attach the data in R
   % # The sep indiciates that they are comma separated
   % # the header option indicates that the first line
   % # contains the variable names
   % 
   % url = 'http://stats191.stanford.edu/data/heights.table'
   % heights <- read.table(url, sep=',', header=T)
   % 
   % # Tell R to put WIFE and HUSBAND in R's toplevel namespace
   % attach(heights)
   % 
   % # Fit simple linear regression model
   % height.lm = lm(WIFE ~ HUSBAND)
   % 
   % # Plot the data
   % plot(HUSBAND, WIFE, pch=23, bg='red', cex=2)
   % 
   % # Add a line to the plot indicating the best
   % # fitting line
   % 
   % abline(height.lm, lwd=3, col='yellow')
   % 
   % 


   \begin{frame}
   \frametitle{Heights data}
   \begin{center}
   \resizebox{!}{2.7in}{\includegraphics{./images/inline/453b25bc12.png}}    
   \end{center}
   \href{http://stats191.stanford.edu/review.html}{R code}
   \end{frame}

   %%%%%%%%%%%%%%%%%%%%%%%%%%%%%%%%%%%%%%%%%%%%%%%%%%%%%%%%%%%%

   \begin{frame} \frametitle{Heights data}

   \begin{block}
   {Linear regression models}
   \begin{itemize}

   \item We might model the data as
   $$
   {\tt Wife} = \B{0} + \B{1} {\tt Husband} + \varepsilon.
   $$

   \item This model is {\em linear} in {\tt Husband}, it is a
   {\em simple linear regression model}.

   \item Another model:
   $$
   {\tt Wife} = \B{0} + \B{1} {\tt Husband} + \B{2} {\tt Mother} + \B{3}
   {\tt Father} + \varepsilon.
   $$

   \item Also linear (in {\tt Husband}, {\tt Mother}, {\tt Father}).

   \item Which model is better? We need a tool to compare models \dots
   \end{itemize}
   \end{block}
   \end{frame}

   %%%%%%%%%%%%%%%%%%%%%%%%%%%%%%%%%%%%%%%%%%%%%%%%%%%%%%%%%%%%

   \begin{frame} \frametitle{Right-to-work example}

   \begin{block}
   {\href{http://www.ilr.cornell.edu/~hadi/RABE4/Data4/P005.txt}{Data} description}
   \begin{itemize}
   \item Income: income for a four-person family
   \item COL: cost of living for a four-person family
   \item PD: Population density
   \item URate: rate of unionization in 1978
   \item Pop: Population
   \item Taxes: Property taxes in 1972
   \item RTWL: right-to-work indicator
   \end{itemize}
   \end{block}
   \end{frame}

   %CODE
       % url = "http://www.ilr.cornell.edu/~hadi/RABE4/Data4/P005.txt"
   % rtw.table <- read.table(url, header=T, sep='\t')
   % attach(rtw.table)
   % boxplot(COL ~ RTWL, col='orange', pch=23, bg='red')


   \begin{frame}
   \frametitle{Right-to-work vs. cost of living}
   \begin{center}
   \resizebox{!}{2.7in}{\includegraphics{./images/inline/25d3613aca.png}}    
   \end{center}
   \href{http://stats191.stanford.edu/review.html}{R code}
   \end{frame}

   %CODE
       % boxplot(Income ~ RTWL, col='orange', pch=23, bg='red')


   \begin{frame}
   \frametitle{Right-to-work vs. income}
   \begin{center}
   \resizebox{!}{2.7in}{\includegraphics{./images/inline/c3f8116055.png}}    
   \end{center}
   \href{http://stats191.stanford.edu/review.html}{R code}
   \end{frame}

   %CODE
       % plot(URate, COL, pch=23, bg='red')


   \begin{frame}
   \frametitle{Unionization vs. cost of living}
   \begin{center}
   \resizebox{!}{2.7in}{\includegraphics{./images/inline/b2e8bbe568.png}}    
   \end{center}
   \href{http://stats191.stanford.edu/review.html}{R code}
   \end{frame}

   %CODE
       % plot(URate, Income, pch=23, bg='red')


   \begin{frame}
   \frametitle{Unionization vs. income}
   \begin{center}
   \resizebox{!}{2.7in}{\includegraphics{./images/inline/ed40338846.png}}    
   \end{center}
   \href{http://stats191.stanford.edu/review.html}{R code}
   \end{frame}

   %CODE
       % plot(URate, Income, pch=23, bg='red')


   \begin{frame}
   \frametitle{Unionization vs. income}
   \begin{center}
   \resizebox{!}{2.7in}{\includegraphics{./images/inline/ed40338846.png}}    
   \end{center}
   \href{http://stats191.stanford.edu/review.html}{R code}
   \end{frame}

   %CODE
       % plot(URate, Pop, pch=23, bg='red')


   \begin{frame}
   \frametitle{Unionization vs. population}
   \begin{center}
   \resizebox{!}{2.7in}{\includegraphics{./images/inline/5cf7ce0169.png}}    
   \end{center}
   \href{http://stats191.stanford.edu/review.html}{R code}
   \end{frame}

   %CODE
       % plot(COL, Income, pch=23, bg='red')


   \begin{frame}
   \frametitle{Cost-of-living vs. income}
   \begin{center}
   \resizebox{!}{2.7in}{\includegraphics{./images/inline/cc7f27df0a.png}}    
   \end{center}
   \href{http://stats191.stanford.edu/review.html}{R code}
   \end{frame}

   %CODE
       % pairs(rtw.table, pch=23, bg='red')


   \begin{frame}
   \frametitle{Full dataset}
   \begin{center}
   \resizebox{!}{2.7in}{\includegraphics{./images/inline/9bcf84dca6.png}}    
   \end{center}
   \href{http://stats191.stanford.edu/review.html}{R code}
   \end{frame}

   %CODE
       % pairs(rtw.table[-27,], pch=23, bg='red')


   \begin{frame}
   \frametitle{Without NYC}
   \begin{center}
   \resizebox{!}{2.7in}{\includegraphics{./images/inline/cc0169edee.png}}    
   \end{center}
   \href{http://stats191.stanford.edu/review.html}{R code}
   \end{frame}

   %%%%%%%%%%%%%%%%%%%%%%%%%%%%%%%%%%%%%%%%%%%%%%%%%%%%%%%%%%%%

   \begin{frame} \frametitle{Right-to-work example}

   \begin{block}
   {Building a model}
   Some of the main goals of this course:

   \begin{itemize}
   \item Build a statistical model describing the ``effect of RTWL'' on ``COL''

   \item This model should recognize that other variables also affect ``COL''

   \item What sort of ``statistical confidence'' do we have in our
   conclusion about ``RTWL'' and ``COL''?

   \item Is the model adequate do describe this dataset?

   \item Are there other (simpler, more complicated) models?
   \end{itemize}
   \end{block}
   \end{frame}

   %%%%%%%%%%%%%%%%%%%%%%%%%%%%%%%%%%%%%%%%%%%%%%%%%%%%%%%%%%%%

   \begin{frame} \frametitle{Descriptive statistics -- numerical}

   \begin{block}
   {Mean of a sample}
   Given a sample of numbers $X=(X_1, \dots, X_n)$ the sample mean,
   $\overline{X}$ is
   $$
   \overline{X} = \frac1n \sum_{i=1}^n X_i.$$
   \end{block}

   \begin{block}
   {Standard deviation of a sample}
   Given a sample of numbers $X=(X_1, \dots, X_n)$ the sample
   standard deviation $S_X$ is
   $$
   S^2_X = \frac{1}{n-1}  \sum_{i=1}^n (X_i-\overline{X})^2.$$
   \end{block}
   \end{frame}

   %%%%%%%%%%%%%%%%%%%%%%%%%%%%%%%%%%%%%%%%%%%%%%%%%%%%%%%%%%%%

   \begin{frame} \frametitle{Descriptive statistics -- numerical}

   \begin{block}
   {Median of a sample}
   Given a sample of numbers $X=(X_1, \dots, X_n)$ the sample median is
   the ``middle'' of the sample:
   if $n$ is even, it is the average of the middle two points.
   If $n$ is odd, it is the midpoint.
   \end{block}

   \begin{block}
   {Quantiles of a sample}
   Given a sample of numbers $X=(X_1, \dots, X_n)$ the  $q$-th quantile is
   a point $x_q$ in the data such that $q \cdot 100\%$ of the data lie to the
   left of $x_q$.

   {\bf Example:} the $0.5$-quantile is the median: half
   of the data lie to the right of the median.
   \end{block}
   \end{frame}

   %CODE
       % hist(treated, main='', xlab='Decrease', col='orange')


   \begin{frame}
   \frametitle{Histogram}
   \begin{center}
   \resizebox{!}{2.7in}{\includegraphics{./images/inline/f79d8915ad.png}}    
   \end{center}
   \href{http://stats191.stanford.edu/review.html}{R code}
   \end{frame}

   %%%%%%%%%%%%%%%%%%%%%%%%%%%%%%%%%%%%%%%%%%%%%%%%%%%%%%%%%%%%

   \begin{frame} \frametitle{Inference about a population mean}

   \begin{block}
   {A testing scenario}
   \begin{itemize}

   \item Suppose we want to determine the efficacy of a
   new drug on blood pressure.

   \item Our study design is: we will treat
   a large patient population with the drug and measure their
   blood pressure before and after taking the drug.

   \item One way to conclude that the drug is effective if the blood pressure has decreased. That is,
   if the average difference is negative.

   \end{itemize}
   \end{block}
   \end{frame}

   %%%%%%%%%%%%%%%%%%%%%%%%%%%%%%%%%%%%%%%%%%%%%%%%%%%%%%%%%%%%

   \begin{frame} \frametitle{Testing hypotheses}

   \begin{block}
   {Setting up the test}
   \begin{itemize}

   \item We could set this up as drawing from a box of {\em differences
   in blood pressure}.

   \item The {\em null hypothesis}, $H_0$ is: ``the average difference is greater than zero.''

   \item The {\em alternative hypothesis}, $H_a$, is: ``the average difference is less than zero.''

   \item Sometimes, people will test the alternative, $H_a$: ``the
   average difference is not zero'' and $H_0$: ``the average difference is zero.''

   \item We test the null with observed data by estimating
    the average difference and converting to standardized units.
   \end{itemize}
   \end{block}
   \end{frame}

   %CODE
       % from matplotlib import rc
   % import pylab, numpy as np, sys
   % np.random.seed(0);import random; random.seed(0)
   % sys.path.append('/private/var/folders/dq/4_9bwd013ln6vvf_q110mwrh0000gn/T/tmpLuOCfp')
   % f=pylab.gcf(); f.set_size_inches(8.0,6.0)
   % datadir ='/private/var/folders/dq/4_9bwd013ln6vvf_q110mwrh0000gn/T/tmpLuOCfp/data'
   % import random
   % X = np.mgrid[0:1:10j,0:1:5j].reshape((2,50)) + np.random.sample((2,50)) * 0.05
   % X = X.T
   % sample = [random.randint(-6,3) for _  in range(50)]
   % for i in range(50):
   %     pylab.text(X[i,0], X[i,1], '%d' % sample[i])
   % 
   %     pylab.gca().set_xticks([]);    pylab.gca().set_xlim([-0.1,1.1])
   %     pylab.gca().set_yticks([]);    pylab.gca().set_ylim([-0.1,1.1])
   % pylab.title(r"$\bar{X}$=average(sample)=%0.1f, $S_X$=SD$^+$(sample)=%0.1f" % (np.mean(sample), np.std(sample) * np.sqrt(50/49.)))
   % 


   \begin{frame}
   \frametitle{Sample of blood pressures}
   \begin{center}
   \resizebox{!}{2.7in}{\includegraphics{./images/inline/eec113e550.pdf}}    
   \end{center}
   Sample of 50
   \end{frame}

   %%%%%%%%%%%%%%%%%%%%%%%%%%%%%%%%%%%%%%%%%%%%%%%%%%%%%%%%%%%%

   \begin{frame} \frametitle{Inference about a population mean}

   \begin{block}
   {Testing whether mean is 0: two-sided}
   \begin{itemize}
   \item  Suppose we want a two-sided test of whether $\mu=0$ based
   on a sample $X$, at level $\alpha$.
   \item Compute
   $$
   T = \frac{\overline{X}}{S_X/\sqrt{n}} = \frac{-0.7}{2.7/\sqrt{50}}=-1.8$$
   \item If $|T| > t_{n-1, 1-\alpha/2}$, then reject $H_0:\mu=0$.
   \item Above, $t_{n-1, 1-\alpha/2}$ is the $1-\frac \alpha 2$ quantile of $t_{n-1}$ random variable, defined by
   $$
   \Pp(T_{n-1} \leq t_{n-1,1-\alpha/2}) = 1 - \frac\alpha 2.$$

   \item With $df=49, \alpha=0.05$, we see that $t_{49,0.975}=2.00$. So,
   we do not reject $H_0$.
   \end{itemize}
   \end{block}
   \end{frame}

   %CODE
       % from matplotlib import rc
   % import pylab, numpy as np, sys
   % np.random.seed(0);import random; random.seed(0)
   % sys.path.append('/private/var/folders/dq/4_9bwd013ln6vvf_q110mwrh0000gn/T/tmpLuOCfp')
   % f=pylab.gcf(); f.set_size_inches(8.0,6.0)
   % datadir ='/private/var/folders/dq/4_9bwd013ln6vvf_q110mwrh0000gn/T/tmpLuOCfp/data'
   % import pylab, numpy as np
   % import scipy.stats
   % 
   % df = 49
   % x = np.linspace(-4,4,101)
   % pylab.plot(x,scipy.stats.t.pdf(x, df)*100, linewidth=2, label=r'$T$, df=49')
   % 
   % # The t region
   % 
   % x2 = np.linspace(scipy.stats.t.isf(0.025,df),4, 101)
   % y2 = scipy.stats.t.pdf(x2, df)
   % xf, yf = pylab.poly_between(x2, 0*x2, y2*100)
   % pylab.fill(xf, yf, facecolor='gray', hatch='\\', alpha=0.5)
   % 
   % x2 = np.linspace(-4,-scipy.stats.t.isf(0.025,df), 101)
   % y2 = scipy.stats.t.pdf(x2, df)
   % xf, yf = pylab.poly_between(x2, 0*x2, y2*100)
   % pylab.fill(xf, yf, facecolor='gray', hatch='\\', alpha=0.5)
   % 
   % pylab.gca().set_xlabel('standardized units')
   % pylab.gca().set_ylabel('% per standardized unit')
   % pylab.legend()
   % #pylab.gca().set_xlim([-2,4])
   % #pylab.gca().set_yticks([])
   % 


   \begin{frame}
   \frametitle{Student's $T$}
   \begin{center}
   \resizebox{!}{2.7in}{\includegraphics{./images/inline/4b49f343db.pdf}}    
   \end{center}
   Two-sided {\color{blue} 5\% rejection rule}, df=49
   \end{frame}

   %%%%%%%%%%%%%%%%%%%%%%%%%%%%%%%%%%%%%%%%%%%%%%%%%%%%%%%%%%%%

   \begin{frame} \frametitle{Inference about a population mean}

   \begin{block}
   {Testing whether mean is $<$ 0: one-sided}
   \begin{itemize}
   \item  Suppose we want a one-sided test of whether $\mu < 0$ based
   on a sample $X$, at level $\alpha$.
   \item For this test, the {\em null} is $H_0:\mu \geq 0$ and
   the alternative is $H_a: \mu < 0$.
   \item Compute
   $$
   T = \frac{\overline{X}}{S_X/\sqrt{n}} = \frac{-0.7}{2.7/\sqrt{50}}=-1.8$$
   \item If $T < t_{n-1, \alpha}$, then reject $H_0:\mu=0$.
   \item With $df=49, \alpha=0.05$, we see that $t_{49,0.05}=-1.68$. So,
   we reject $H_0$.
   \end{itemize}
   \end{block}
   \end{frame}

   %CODE
       % from matplotlib import rc
   % import pylab, numpy as np, sys
   % np.random.seed(0);import random; random.seed(0)
   % sys.path.append('/private/var/folders/dq/4_9bwd013ln6vvf_q110mwrh0000gn/T/tmpLuOCfp')
   % f=pylab.gcf(); f.set_size_inches(8.0,6.0)
   % datadir ='/private/var/folders/dq/4_9bwd013ln6vvf_q110mwrh0000gn/T/tmpLuOCfp/data'
   % import pylab, numpy as np
   % import scipy.stats
   % 
   % df = 49
   % x = np.linspace(-4,4,101)
   % pylab.plot(x,scipy.stats.t.pdf(x, df)*100, linewidth=2, label=r'$T$, df=49')
   % 
   % # The t region
   % 
   % #x2 = np.linspace(scipy.stats.t.isf(0.025,df),4, 101)
   % #y2 = scipy.stats.t.pdf(x2, df)
   % #xf, yf = pylab.poly_between(x2, 0*x2, y2*100)
   % #pylab.fill(xf, yf, facecolor='gray', hatch='\\', alpha=0.5)
   % 
   % x2 = np.linspace(-4,-scipy.stats.t.isf(0.05,df), 101)
   % y2 = scipy.stats.t.pdf(x2, df)
   % xf, yf = pylab.poly_between(x2, 0*x2, y2*100)
   % pylab.fill(xf, yf, facecolor='gray', hatch='\\', alpha=0.5)
   % 
   % pylab.gca().set_xlabel('standardized units')
   % pylab.gca().set_ylabel('% per standardized unit')
   % pylab.legend()
   % #pylab.gca().set_xlim([-2,4])
   % #pylab.gca().set_yticks([])
   % 


   \begin{frame}
   \frametitle{Student's $T$}
   \begin{center}
   \resizebox{!}{2.7in}{\includegraphics{./images/inline/e101a35f76.pdf}}    
   \end{center}
   One-sided {\color{blue} 5\% rejection rule} for $H_0:\mu \geq 0$, df=49
   \end{frame}

   %CODE
       % from matplotlib import rc
   % import pylab, numpy as np, sys
   % np.random.seed(0);import random; random.seed(0)
   % sys.path.append('/private/var/folders/dq/4_9bwd013ln6vvf_q110mwrh0000gn/T/tmpLuOCfp')
   % f=pylab.gcf(); f.set_size_inches(8.0,6.0)
   % datadir ='/private/var/folders/dq/4_9bwd013ln6vvf_q110mwrh0000gn/T/tmpLuOCfp/data'
   % import pylab, numpy as np
   % import scipy.stats
   % 
   % df = 5
   % x = np.linspace(-4,4,101)
   % pylab.plot(x,scipy.stats.t.pdf(x, df)*100, linewidth=2, label=r'$T$, df=5')
   % 
   % # The t region
   % 
   % #x2 = np.linspace(scipy.stats.t.isf(0.025,df),4, 101)
   % #y2 = scipy.stats.t.pdf(x2, df)
   % #xf, yf = pylab.poly_between(x2, 0*x2, y2*100)
   % #pylab.fill(xf, yf, facecolor='gray', hatch='\\', alpha=0.5)
   % 
   % x2 = np.linspace(-4,-scipy.stats.t.isf(0.05,df), 101)
   % y2 = scipy.stats.t.pdf(x2, df)
   % xf, yf = pylab.poly_between(x2, 0*x2, y2*100)
   % pylab.fill(xf, yf, facecolor='gray', hatch='\\', alpha=0.5)
   % 
   % pylab.gca().set_xlabel('standardized units')
   % pylab.gca().set_ylabel('% per standardized unit')
   % pylab.legend()
   % #pylab.gca().set_xlim([-2,4])
   % #pylab.gca().set_yticks([])
   % 


   \begin{frame}
   \frametitle{Student's $T$}
   \begin{center}
   \resizebox{!}{2.7in}{\includegraphics{./images/inline/a31bc72a1c.pdf}}    
   \end{center}
   One-sided {\color{blue} 5\% rejection rule} for $H_0:\mu \geq 0$, df=5
   \end{frame}

   %%%%%%%%%%%%%%%%%%%%%%%%%%%%%%%%%%%%%%%%%%%%%%%%%%%%%%%%%%%%

   \begin{frame} \frametitle{Inference about a population mean}

   \begin{block}
   {Confidence interval}
   \begin{itemize}
   \item     If $(X_1, \dots, X_n)$ are independent, all having a normal distribution  $N(\mu, \sigma^2)$, then a $(1 - \alpha)$-confidence interval for $\mu$ is
   $$
   \overline{ X} \pm t_{n-1, 1 - \alpha/2}\cdot S_X / \sqrt{n}
   $$
   \item That is, if $\alpha=0.05$, and we repeat the experiment
   many times then 95\% of the time,
   the true $\mu$ will be in the interval
   $$
   [\overline{ X} - t_{n-1, 1 - \alpha/2}\cdot S_X / \sqrt{n},\overline{ X} + t_{n-1, 1 - \alpha/2}\cdot S_X / \sqrt{n}]
   $$
   \item Again, $t_{n-1, 1-\alpha/2}$ is the $1-\frac \alpha 2$ quantile of $t_{n-1}$ random variable, defined by
   $$
   \Pp(T_{n-1} \leq t_{n-1,1-\alpha/2}) = 1 - \frac\alpha 2.$$
   \end{itemize}
   \end{block}
   \end{frame}

   %CODE
       % mu = 2
   % nsample <- 500 # how many samples to generate
   % nobs <- 10  # how many observations in each sample
   % alpha <- 0.15
   % 
   % CI.data <- matrix(0, nrow=nsample, ncol=2) # a matrix to store the data
   % 
   % cover <- numeric(nsample) # a counter to see how many CIs contain 0
   % for (i in 1:nsample) {
   %   CI.data[i,] <- t.test(rnorm(nobs) + mu, conf.level=1-alpha)$conf.int
   %   cover[i] <- (CI.data[i,1] < mu) * (CI.data[i,2] > mu) # add 1 if
   %                                         # CI contains 0
   % }
   % print(sum(cover)/nsample) # coverage percentage, should be approx 1-alpha
   % 
   % simulate = function() {
   % nplot <- 20 # how many intervals to plot
   % 
   % plot(c(-2+mu,2+mu), c(1, nplot), type='n', xlab='Confidence Intervals', ylab='Sample')
   % for (i in 1:min(nsample, 20)) {
   %   if (cover[i]) {
   %     lines(CI.data[i,], rep(i,2), col='red', lwd=2) # add a red bar for
   %                                         # each CI that covers
   %   }
   %   else {
   %     lines(CI.data[i,], rep(i,2), col='blue', lwd=2) # add a red bar for
   %                                         # each CI that covers
   %   }
   % }
   % }
   % simulate()


   \begin{frame}
   \frametitle{20 different confidence 85\% intervals}
   \begin{center}
   \resizebox{!}{2.7in}{\includegraphics{./images/inline/ef87c09474.png}}    
   \end{center}

   \end{frame}

   %CODE
       % mu = 2
   % x=3
   % nsample <- 500 # how many samples to generate
   % nobs <- 10  # how many observations in each sample
   % alpha <- 0.15
   % 
   % CI.data <- matrix(0, nrow=nsample, ncol=2) # a matrix to store the data
   % 
   % cover <- numeric(nsample) # a counter to see how many CIs contain 0
   % for (i in 1:nsample) {
   %   CI.data[i,] <- t.test(rnorm(nobs) + mu, conf.level=1-alpha)$conf.int
   %   cover[i] <- (CI.data[i,1] < mu) * (CI.data[i,2] > mu) # add 1 if
   %                                         # CI contains 0
   % }
   % print(sum(cover)/nsample) # coverage percentage, should be approx 1-alpha
   % 
   % simulate = function() {
   % nplot <- 20 # how many intervals to plot
   % 
   % plot(c(-2+mu,2+mu), c(1, nplot), type='n', xlab='Confidence Intervals', ylab='Sample')
   % for (i in 1:min(nsample, 20)) {
   %   if (cover[i]) {
   %     lines(CI.data[i,], rep(i,2), col='red', lwd=2) # add a red bar for
   %                                         # each CI that covers
   %   }
   %   else {
   %     lines(CI.data[i,], rep(i,2), col='blue', lwd=2) # add a red bar for
   %                                         # each CI that covers
   %   }
   % }
   % }
   % simulate()


   \begin{frame}
   \frametitle{Another 20}
   \begin{center}
   \resizebox{!}{2.7in}{\includegraphics{./images/inline/ae224efefc.png}}    
   \end{center}

   \end{frame}

   %CODE
       % mu = 2
   % x=4
   % nsample <- 500 # how many samples to generate
   % nobs <- 10  # how many observations in each sample
   % alpha <- 0.15
   % 
   % CI.data <- matrix(0, nrow=nsample, ncol=2) # a matrix to store the data
   % 
   % cover <- numeric(nsample) # a counter to see how many CIs contain 0
   % for (i in 1:nsample) {
   %   CI.data[i,] <- t.test(rnorm(nobs) + mu, conf.level=1-alpha)$conf.int
   %   cover[i] <- (CI.data[i,1] < mu) * (CI.data[i,2] > mu) # add 1 if
   %                                         # CI contains 0
   % }
   % print(sum(cover)/nsample) # coverage percentage, should be approx 1-alpha
   % 
   % simulate = function() {
   % nplot <- 20 # how many intervals to plot
   % 
   % plot(c(-2+mu,2+mu), c(1, nplot), type='n', xlab='Confidence Intervals', ylab='Sample')
   % for (i in 1:min(nsample, 20)) {
   %   if (cover[i]) {
   %     lines(CI.data[i,], rep(i,2), col='red', lwd=2) # add a red bar for
   %                                         # each CI that covers
   %   }
   %   else {
   %     lines(CI.data[i,], rep(i,2), col='blue', lwd=2) # add a red bar for
   %                                         # each CI that covers
   %   }
   % }
   % }
   % simulate()


   \begin{frame}
   \frametitle{Yet another 20}
   \begin{center}
   \resizebox{!}{2.7in}{\includegraphics{./images/inline/fb88fdd606.png}}    
   \end{center}

   \end{frame}

   %CODE
       % mu = 2
   % x=5
   % nsample <- 500 # how many samples to generate
   % nobs <- 10  # how many observations in each sample
   % alpha <- 0.15
   % 
   % CI.data <- matrix(0, nrow=nsample, ncol=2) # a matrix to store the data
   % 
   % cover <- numeric(nsample) # a counter to see how many CIs contain 0
   % for (i in 1:nsample) {
   %   CI.data[i,] <- t.test(rnorm(nobs) + mu, conf.level=1-alpha)$conf.int
   %   cover[i] <- (CI.data[i,1] < mu) * (CI.data[i,2] > mu) # add 1 if
   %                                         # CI contains 0
   % }
   % print(sum(cover)/nsample) # coverage percentage, should be approx 1-alpha
   % 
   % simulate = function() {
   % nplot <- 20 # how many intervals to plot
   % 
   % plot(c(-2+mu,2+mu), c(1, nplot), type='n', xlab='Confidence Intervals', ylab='Sample')
   % for (i in 1:min(nsample, 20)) {
   %   if (cover[i]) {
   %     lines(CI.data[i,], rep(i,2), col='red', lwd=2) # add a red bar for
   %                                         # each CI that covers
   %   }
   %   else {
   %     lines(CI.data[i,], rep(i,2), col='blue', lwd=2) # add a red bar for
   %                                         # each CI that covers
   %   }
   % }
   % }
   % simulate()


   \begin{frame}
   \frametitle{Yes, 20 more}
   \begin{center}
   \resizebox{!}{2.7in}{\includegraphics{./images/inline/6f9a4bcdc4.png}}    
   \end{center}

   \end{frame}

   %CODE
       % mu = 2
   % x=6
   % nsample <- 500 # how many samples to generate
   % nobs <- 10  # how many observations in each sample
   % alpha <- 0.15
   % 
   % CI.data <- matrix(0, nrow=nsample, ncol=2) # a matrix to store the data
   % 
   % cover <- numeric(nsample) # a counter to see how many CIs contain 0
   % for (i in 1:nsample) {
   %   CI.data[i,] <- t.test(rnorm(nobs) + mu, conf.level=1-alpha)$conf.int
   %   cover[i] <- (CI.data[i,1] < mu) * (CI.data[i,2] > mu) # add 1 if
   %                                         # CI contains 0
   % }
   % print(sum(cover)/nsample) # coverage percentage, should be approx 1-alpha
   % 
   % simulate = function() {
   % nplot <- 20 # how many intervals to plot
   % 
   % plot(c(-2+mu,2+mu), c(1, nplot), type='n', xlab='Confidence Intervals', ylab='Sample')
   % for (i in 1:min(nsample, 20)) {
   %   if (cover[i]) {
   %     lines(CI.data[i,], rep(i,2), col='red', lwd=2) # add a red bar for
   %                                         # each CI that covers
   %   }
   %   else {
   %     lines(CI.data[i,], rep(i,2), col='blue', lwd=2) # add a red bar for
   %                                         # each CI that covers
   %   }
   % }
   % }
   % simulate()


   \begin{frame}
   \frametitle{A final 20}
   \begin{center}
   \resizebox{!}{2.7in}{\includegraphics{./images/inline/b543efd3af.png}}    
   \end{center}
   Out of 100, 90 covered the true mean...
   \end{frame}

   %%%%%%%%%%%%%%%%%%%%%%%%%%%%%%%%%%%%%%%%%%%%%%%%%%%%%%%%%%%%

   \begin{frame} \frametitle{Review}

   \begin{block}
   {Effect of calcium on BP}
   \begin{itemize}
   \item A study was conducted to study the effect of calcium supplements
   on blood pressure.
   \item More detailed data description can be found
   \href{http://lib.stat.cmu.edu/DASL/Datafiles/Calcium.html}{\mycolor{here}}.
   \end{itemize}
   \end{block}

   \begin{block}
   {Questions}
   \begin{itemize}
   \item What is the mean decrease in BP in the treated group? placebo group?
   \item What is the median decrease in BP in the treated group? placebo group?
   \item What is the standard deviation of decrease in BP in the treated group? placebo group?
   \item Is there a difference between the two groups? Did BP decrease more in the treated group?
   \end{itemize}
   \end{block}
   \end{frame}

   %CODE
       % url = 'http://lib.stat.cmu.edu/DASL/Datafiles/Calcium.html'
   % calcium.table <- read.table(url,header=T,skip=28,nrow=21)
   % 
   % # Attach the table so R can find the variables Decrease and Treatment
   % 
   % attach(calcium.table)
   % 
   % # Numerical summaries of the two groups
   % 
   % treated <- Decrease[(Treatment == 'Calcium')]
   % placebo <- Decrease[(Treatment == 'Placebo')]
   % 
   % boxplot(Decrease ~ Treatment, col='orange', pch=23, bg='red')


   \begin{frame}
   \frametitle{Boxplot}
   \begin{center}
   \resizebox{!}{2.7in}{\includegraphics{./images/inline/427bf56cd4.png}}    
   \end{center}
   \href{http://stats191.stanford.edu/review.html}{R code}
   \end{frame}

   %CODE
       % hist(treated, main='', xlab='Decrease', col='orange')


   \begin{frame}
   \frametitle{Histogram of Treated response}
   \begin{center}
   \resizebox{!}{2.7in}{\includegraphics{./images/inline/f79d8915ad.png}}    
   \end{center}
   \href{http://stats191.stanford.edu/review.html}{R code}
   \end{frame}

   %CODE
       % hist(placebo, main='', xlab='Decrease', col='orange')


   \begin{frame}
   \frametitle{Histogram of Placebo response}
   \begin{center}
   \resizebox{!}{2.7in}{\includegraphics{./images/inline/b1fb4584c5.png}}    
   \end{center}
   \href{http://stats191.stanford.edu/review.html}{R code}
   \end{frame}

   %%%%%%%%%%%%%%%%%%%%%%%%%%%%%%%%%%%%%%%%%%%%%%%%%%%%%%%%%%%%

   \begin{frame} \frametitle{Difference between means}

   \begin{block}
   {BP example}
   \begin{itemize}
   \item In our setting, we have two groups that we have reason to believe are different.
   \item We have two samples:
   \begin{enumerate}
   \item $(X_1, \dots, X_{10})$ (Calcium)
   \item  $(Z_1, \dots, Z_{11})$ (Placebo)
   \end{enumerate}
   \item Does treatment have an effect?
   \item We can answer this statistically by testing the null hypothesis  $H_0:\mu_X = \mu_Z$?
   \end{itemize}
   \end{block}
   \end{frame}

   %%%%%%%%%%%%%%%%%%%%%%%%%%%%%%%%%%%%%%%%%%%%%%%%%%%%%%%%%%%%

   \begin{frame} \frametitle{Difference between means}

   \begin{block}
   {Testing $H_0:\mu_X=\mu_Z$}
   \begin{itemize}
   \item If variances are assumed equal, pooled $t$-test is appropriate
   $$
   T = \frac{\overline{X} - \overline{Z}}{S_P \sqrt{\frac{1}{10}
   + \frac{1}{11}}}, \qquad S^2_P = \frac{9 \cdot S^2_X + 10 \cdot S^2_Z}{19}.$$
   \item For two-sided test at level $\alpha=0.05$, reject if $|T| > t_{19, 0.975}$.
   \item Confidence interval: for example, a $90\%$ confidence interval
   for $\mu_X-\mu_Z$ is
   $$
   \overline{X}-\overline{Z} \pm S_P \sqrt{\frac{1}{10}
   + \frac{1}{11}} \cdot  t_{19,0.95}
   $$
   \end{itemize}
   \end{block}
   \end{frame}

   %%%%%%%%%%%%%%%%%%%%%%%%%%%%%%%%%%%%%%%%%%%%%%%%%%%%%%%%%%%%

   \begin{frame} \frametitle{Difference between means}

   \begin{block}
   {Pooled estimate $S_P$}
   \begin{itemize}
   \item The rule for the $SD$ of differences is
   $$
   SD(\overline{X}-\overline{Z}) = \sqrt{SD(\overline{X})^2+SD(\overline{Z})^2}.
   $$
   \item By this rule, we might take our estimate to be
   $$
   \widehat{SD(\overline{X}-\overline{Z})} = \sqrt{\frac{S^2_X}{10} + \frac{S^2_Z}{11}}
   $$
   \item But, the pooled estimate assumes $\Ee(S^2_X)=\Ee(S^2_Z)=\sigma^2$ and replaces
   the $S^2$'s above with $S^2_P$, a ``better'' estimate of
   $\sigma^2$ than either $S^2_X$ or $S^2_Z$.
   \end{itemize}
   \end{block}
   \end{frame}

   %%%%%%%%%%%%%%%%%%%%%%%%%%%%%%%%%%%%%%%%%%%%%%%%%%%%%%%%%%%%

   \begin{frame} \frametitle{Difference between means}

   \begin{block}
   {Pooled estimate $S_P$}
   \begin{itemize}
   \item Where do we get 19 degrees of freedom?
   \item Well, the $X$ ({\tt Treatment}) sample has $10-1=9$ degrees of freedom
   to estimate $\sigma^2$ while the $Z$ ({\tt Placebo}) sample
   has $11-1=10$ degrees of freedom.
   \item The total degrees of freedom is $9+10=19$.
   \end{itemize}
   \end{block}
   \end{frame}

   %%%%%%%%%%%%%%%%%%%%%%%%%%%%%%%%%%%%%%%%%%%%%%%%%%%%%%%%%%%%

   \begin{frame} \frametitle{Our first regression model}

   \begin{block}
   {Unified dataset}
   \begin{itemize}

   \item Put two samples together:
   $$Y=(X_1,\dots, X_{10}, Z_1, \dots, Z_{11}).$$

   \item Under the same assumptions as the pooled $t$-test:
   $$
   \begin{aligned}
   Y_i &\sim N(\mu_i, \sigma^2)\\
   \mu_i &=
   \begin{cases}
   \mu_X & 1 \leq i \leq 10 \\ \mu_Z & 11 \leq i \leq 21
   \end{cases}
   \end{aligned}
   $$
   \end{itemize}
   \end{block}
   \end{frame}

   %%%%%%%%%%%%%%%%%%%%%%%%%%%%%%%%%%%%%%%%%%%%%%%%%%%%%%%%%%%%

   \begin{frame} \frametitle{Our first regression model}

   \begin{block}
    {$t$-test as regression model}
   \begin{itemize}
   \item This is a (regression) model for the sample $Y$. The
   (qualitative) variable \Rscode{Treatment} is
   called a ``covariate'' or ``predictor''.
   \item The decrease in BP is an outcome variable.
   \item We assume that the relationship between treatment and average
   decrease in BP is simple: it depends only on which group a subject is in.
   \item This relationship is ``modelled'' through the mean
   vector $\mu=(\mu_1, \dots, \mu_{21})$.
   \end{itemize}
   \end{block}
   \end{frame}

   %%%%%%%%%%%%%%%%%%%%%%%%%%%%%%%%%%%%%%%%%%%%%%%%%%%%%%%%%%%%

   \begin{frame} 

   \end{frame}

   \end{document}
