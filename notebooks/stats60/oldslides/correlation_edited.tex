   \documentclass[handout]{beamer}



   \mode<presentation>
   {
     \usetheme{PaloAlto}
   \setbeamertemplate{footline}[page number]

     \setbeamercolor{sidebar}{bg=white, fg=black}
     \setbeamercolor{frametitle}{bg=white, fg=black}
     % or ...
     \setbeamercolor{logo}{bg=white}
     \setbeamercolor{block body}{parent=normal text,bg=white}
     \setbeamercolor{author in sidebar}{fg=black}
     \setbeamercolor{title in sidebar}{fg=black}


     \setbeamercolor*{block title}{use=structure,fg=structure.fg,bg=structure.fg!20!bg}
     \setbeamercolor*{block title alerted}{use=alerted text,fg=alerted text.fg,bg=alerted text.fg!20!bg}
     \setbeamercolor*{block title example}{use=example text,fg=example text.fg,bg=example text.fg!20!bg}


     \setbeamercolor{block body}{parent=normal text,use=block title,bg=block title.bg!50!bg}
     \setbeamercolor{block body alerted}{parent=normal text,use=block title alerted,bg=block title alerted.bg!50!bg}
     \setbeamercolor{block body example}{parent=normal text,use=block title example,bg=block title example.bg!50!bg}

     % or ...

     \setbeamercovered{transparent}
     % or whatever (possibly just delete it)
     \logo{\resizebox{!}{1.5cm}{\href{\basename{R}}{\includegraphics{image}}}}
   }

   \mode<handout>
   {
     \usetheme{PaloAlto}
     \usecolortheme{default}
     \setbeamercolor{sidebar}{bg=white, fg=black}
     \setbeamercolor{frametitle}{bg=white, fg=black}
     % or ...
     \setbeamercolor{logo}{bg=white}
     \setbeamercolor{block body}{parent=normal text,bg=white}
     \setbeamercolor{author in sidebar}{fg=black}
     \setbeamercolor{title in sidebar}{fg=black}
     \setbeamercovered{transparent}
     % or whatever (possibly just delete it)
     \logo{}
   }

   \usepackage{epsdice}
   \usepackage[latin1]{inputenc}
   \usepackage{graphics}
   \usepackage{amsmath,eepic,epic}

   \newcommand{\figslide}[3]{
   \begin{frame}
   \frametitle{#1}
     \begin{center}
     \resizebox{!}{2.7in}{\includegraphics{#2}}    
     \end{center}
   {#3}
   \end{frame}
   }

   \newcommand{\fighslide}[4]{
   \begin{frame}
   \frametitle{#1}
     \begin{center}
     \resizebox{!}{#4}{\includegraphics{#2}}    
     \end{center}
   {#3}
   \end{frame}
   }

   \newcommand{\figwref}[1]{
   \href{#1}{\tiny \tt #1}}

   \newcommand{\B}[1]{\beta_{#1}}
   \newcommand{\Bh}[1]{\widehat{\beta}_{#1}}
   \newcommand{\V}{\text{Var}}
   \newcommand{\Cov}{\text{Cov}}
   \newcommand{\Vh}{\widehat{\V}}
   \newcommand{\s}{\sigma}
   \newcommand{\sh}{\widehat{\sigma}}

   \newcommand{\argmax}[1]{\mathop{\text{argmax}}_{#1}}
   \newcommand{\argmin}[1]{\mathop{\text{argmin}}_{#1}}
   \newcommand{\Ee}{\mathbb{E}}
   \newcommand{\Pp}{\mathbb{P}}
   \newcommand{\real}{\mathbb{R}}
   \newcommand{\Ybar}{\overline{Y}}
   \newcommand{\Yh}{\widehat{Y}}
   \newcommand{\Xbar}{\overline{X}}
   \newcommand{\Tr}{\text{Tr}}


   \newcommand{\model}{{\cal M}}

   \newcommand{\figvskip}{-0.7in}
   \newcommand{\fighskip}{-0.3in}
   \newcommand{\figheight}{3.5in}

   \newcommand{\Rcode}[1]{{\bf \tt #1 }}
   \newcommand{\Rtcode}[1]{{\tiny \bf \tt #1 }}
   \newcommand{\Rscode}[1]{{\small \bf \tt #1 }}

   \newcommand{\RR}{{\tt R} \;}
   \newcommand{\basename}[1]{http://stats60.stanford.edu/#1}
   \newcommand{\dataname}[1]{\basename{data/#1}}
   \newcommand{\Rname}[1]{\basename{R/#1}}

   \newcommand{\mycolor}[1]{{\color{blue} #1}}
   \newcommand{\basehref}[2]{\href{\basename{#1}}{\mycolor{#2}}}
   \newcommand{\Rhref}[2]{\href{\basename{R/#1}}{\mycolor{#2}}}
   \newcommand{\datahref}[2]{\href{\dataname{#1}}{\mycolor{#2}}}
   \newcommand{\X}{\pmb{X}}
   \newcommand{\Y}{\pmb{Y}}
   \newcommand{\be}{\pmb{varepsilon}}
   \newcommand{\logit}{\text{logit}}


   \title{Statistics 60: Introduction to Statistical Methods}
   \subtitle{Chapter 8: Correlation} 
   \author{}% {Jonathan Taylor \\
   %Department of Statistics \\
   %Stanford University
   %}


   \begin{document}

   \begin{frame}
   \titlepage
   \end{frame}

   %CODE
       % from matplotlib import rc
   % import pylab, numpy as np, sys
   % np.random.seed(0);import random; random.seed(0)
   % import matplotlib.mlab as ML
   % H = ML.csv2rec('%s/pearson_lee.csv' % datadir)
   % M = H['mother']
   % D = H['daughter']
   % pylab.scatter(M, D, c='red')
   % pylab.gca().set_xlabel("Mother's height (inches)")
   % pylab.gca().set_ylabel("Daughter's height (inches)")
   % 


   \begin{frame}
   \frametitle{Correlation}
   \begin{center}
   \resizebox{!}{2.7in}{\includegraphics{./images/inline/a84b90b10a.pdf}}    
   \end{center}
   There is a {\em positive association}
   \end{frame}

   %%%%%%%%%%%%%%%%%%%%%%%%%%%%%%%%%%%%%%%%%%%%%%%%%%%%%%%%%%%%

   \begin{frame} \frametitle{Scatterplot}

   \begin{block}
   {Drawing a scatter plot}
   Given two lists: $X, Y$ of length $n$
   \begin{description}
   \item[Coordinate system:] Draw a coordinate system
   between roughly $[\min(X), \max(X)]$ on the $x$-axis and
   between roughly $[\min(Y), \max(Y)]$ on the $y$-axis.

   \item[Plotting points] Each pair $(X_i, Y_i)$ gets a point
   with those corresponding coordinates.
   \end{description}
   \end{block}
   \end{frame}

   %CODE
       % from matplotlib import rc
   % import pylab, numpy as np, sys
   % np.random.seed(0);import random; random.seed(0)
   % pylab.scatter([1,4,6],[-2,2,8], c='r')
   % pylab.gca().set_xlabel('X')
   % pylab.gca().set_ylabel('Y')
   % 


   \begin{frame}
   \frametitle{Scatterplot example: $X=[1,4,6], Y=[-2,2,8]$}
   \begin{center}
   \resizebox{!}{2.7in}{\includegraphics{./images/inline/09a6b8fef0.pdf}}    
   \end{center}

   \end{frame}

   %CODE
       % from matplotlib import rc
   % import pylab, numpy as np, sys
   % np.random.seed(0);import random; random.seed(0)
   % pylab.scatter([1,4,6],[-2,2,8], c='r')
   % pylab.gca().set_xlabel('Independent')
   % pylab.gca().set_ylabel('Dependent')
   % 


   \begin{frame}
   \frametitle{Dependent and independent variables}
   \begin{center}
   \resizebox{!}{2.7in}{\includegraphics{./images/inline/56dc1e54fa.pdf}}    
   \end{center}

   \end{frame}

   %CODE
       % from matplotlib import rc
   % import pylab, numpy as np, sys
   % np.random.seed(0);import random; random.seed(0)
   % import matplotlib.mlab as ML
   % H = ML.csv2rec('%s/pearson_lee.csv' % datadir)
   % M = H['mother']
   % D = H['daughter']
   % pylab.scatter(M, D, c='red')
   % xf, yf = pylab.poly_between([64.5,65.5], [50,50], [75, 75])
   % pylab.fill(xf, yf, facecolor='blue', alpha=0.4, hatch='/')
   % pylab.gca().set_xlabel("Mother's height (inches)")
   % pylab.gca().set_ylabel("Daughter's height (inches)")
   % 


   \begin{frame}
   \frametitle{Correlation}
   \begin{center}
   \resizebox{!}{2.7in}{\includegraphics{./images/inline/ed6c1f2900.pdf}}    
   \end{center}
   What if we want to guess daughter's height when mother = 65 in?
   \end{frame}

   %CODE
       % from matplotlib import rc
   % import pylab, numpy as np, sys
   % np.random.seed(0);import random; random.seed(0)
   % import matplotlib.mlab as ML
   % H = ML.csv2rec('%s/pearson_lee.csv' % datadir)
   % M = H['mother']
   % D = H['daughter']
   % pylab.scatter(M, D, c='red')
   % xf, yf = pylab.poly_between([64.5,65.5], [50,50], [75, 75])
   % pylab.fill(xf, yf, facecolor='blue', alpha=0.4, hatch='/')
   % pylab.gca().set_xlim([64.5,65.5])
   % pylab.gca().set_xlabel("Mother's height (inches)")
   % pylab.gca().set_ylabel("Daughter's height (inches)")
   % 


   \begin{frame}
   \frametitle{Correlation}
   \begin{center}
   \resizebox{!}{2.7in}{\includegraphics{./images/inline/4562d320bd.pdf}}    
   \end{center}
   Still some variability, not completely linear \dots
   \end{frame}

   %%%%%%%%%%%%%%%%%%%%%%%%%%%%%%%%%%%%%%%%%%%%%%%%%%%%%%%%%%%%

   \begin{frame} \frametitle{Correlation}

   \begin{block}
   {Conceptual definition}

   \begin{itemize}
   \item A numerical summary of a scatterplot, i.e. a pair of lists.

   \item    If there is a strong association between two variables, then
   knowing one helps a lot in predicting the other. But when
   there is a weak association, information about one variable
   does not help much in guessing the other.

   \item The {\em correlation coefficient}, $r$ is a measure of the strength of this association.

   \item $r=+1$ if the variables are perfectly positively associated.

   \item $r=-1$ if the variables are perfectly negatively associated.
   \end{itemize}



   \end{block}
   \end{frame}

   %CODE
       % from matplotlib import rc
   % import pylab, numpy as np, sys
   % np.random.seed(0);import random; random.seed(0)
   % X = np.random.standard_normal(50)
   % Y = np.random.standard_normal(50)
   % pylab.scatter(X, X)
   % pylab.gca().set_xticks([])
   % pylab.gca().set_yticks([])
   % pylab.title('r=+1')
   % 


   \begin{frame}
   \frametitle{Perfectly positively correlated}
   \begin{center}
   \resizebox{!}{2.7in}{\includegraphics{./images/inline/eea638aba8.pdf}}    
   \end{center}

   \end{frame}

   %CODE
       % from matplotlib import rc
   % import pylab, numpy as np, sys
   % np.random.seed(0);import random; random.seed(0)
   % X = np.random.standard_normal(50)
   % Y = np.random.standard_normal(50)
   % pylab.scatter(X, -X)
   % pylab.gca().set_xticks([])
   % pylab.gca().set_yticks([])
   % pylab.title('r=-1')
   % 


   \begin{frame}
   \frametitle{Perfectly negatively correlated, $r=-1$}
   \begin{center}
   \resizebox{!}{2.7in}{\includegraphics{./images/inline/f1e21cd2c7.pdf}}    
   \end{center}

   \end{frame}

   %CODE
       % from matplotlib import rc
   % import pylab, numpy as np, sys
   % np.random.seed(0);import random; random.seed(0)
   % X = np.random.standard_normal(50)
   % Y = np.random.standard_normal(50)
   % pylab.scatter(X, Y)
   % pylab.gca().set_xticks([])
   % pylab.gca().set_yticks([])
   % pylab.title('r=0')
   % 


   \begin{frame}
   \frametitle{Uncorrelated variables (no relation) $r=0$}
   \begin{center}
   \resizebox{!}{2.7in}{\includegraphics{./images/inline/1be958ea52.pdf}}    
   \end{center}

   \end{frame}

   %CODE
       % from matplotlib import rc
   % import pylab, numpy as np, sys
   % np.random.seed(0);import random; random.seed(0)
   % def c_from_r(r):
   %    return np.sqrt(r**2 / (1.-r**2)) * np.sign(r)
   % 
   % X = np.random.standard_normal(50)
   % Y = np.random.standard_normal(50)
   % for i, r in zip([1,2,3,4], [0.5,0.9,-0.5,-0.9]):
   %    pylab.subplot(2,2,i)
   %    pylab.scatter(X, Y + c_from_r(r) * X); pylab.gca().set_xticks([]); pylab.gca().set_yticks([]);
   %    pylab.title('r=%0.1f' % r)
   % 


   \begin{frame}
   \frametitle{Positive and negative correlation}
   \begin{center}
   \resizebox{!}{2.7in}{\includegraphics{./images/inline/8c3026698f.pdf}}    
   \end{center}

   \end{frame}

   %%%%%%%%%%%%%%%%%%%%%%%%%%%%%%%%%%%%%%%%%%%%%%%%%%%%%%%%%%%%

   \begin{frame} \frametitle{Correlation}

   \begin{block}
   {Computing $r$, the correlation coefficient}

   \begin{itemize}
   \item Given two lists, $X, Y$, convert them
   each to standardized units. Call these new lists $Z_X, Z_Y$.

   \item Make a new list $Z_{XY}$ whose entries are the products
   of the entries of $Z_X, Z_Y$.
   \item Then,
   $$
   r = \text{average}(Z_{XY}).
   $$
   \item Another way,
   $$
   r = \frac{\text{average(products $X, Y$)} - \text{average}(X) \times \text{average}(Y)}{\text{SD}(X) \times \text{SD}(Y)}.
   $$
   \end{itemize}
   \end{block}
   \end{frame}

   %%%%%%%%%%%%%%%%%%%%%%%%%%%%%%%%%%%%%%%%%%%%%%%%%%%%%%%%%%%%

   \begin{frame} \frametitle{Correlation}

   \begin{block}
   {$\Sigma$ notation}

   \begin{itemize}
   \item The entries of the lists $Z_X, Z_Y, Z_{XY}$ are:
   $$
   \begin{aligned}
   Z_{X,i} &= \frac{X_i - \bar{X}}{\text{SD}(X)} \\
   Z_{Y,i} &= \frac{Y_i - \bar{Y}}{\text{SD}(Y)} \\
   Z_{XY,i} &= Z_{X,i} \times Z_{Y,i}
   \end{aligned}
   $$

   \item Then,
   $$
   r = r(X,Y) = \bar{Z}_{XY} = \frac{1}{n} \sum_{i=1}^n Z_{XY,i}.
   $$
   \end{itemize}

   \end{block}
   \end{frame}

   %%%%%%%%%%%%%%%%%%%%%%%%%%%%%%%%%%%%%%%%%%%%%%%%%%%%%%%%%%%%

   \begin{frame} \frametitle{Correlation}

   \begin{block}
   {$\Sigma$ notation}

   \begin{itemize}
   \item Another way: if $XY$ is the list with entries $X_i \times Y_i$, then
   $$
   r = \frac{\overline{XY} - \bar{X} \times \bar{Y}}{\text{SD}(X) \times \text{SD}(y)}.
   $$
   \end{itemize}

   \end{block}

   \begin{block}
   {Example: $X=[1,4,6], Y=[-2,2,8]$}
   $$
   \begin{aligned}
   \bar{X} &= 3.66 & \text{SD}(X) &= 2.05 \\
   \bar{Y} &= 2.66 & \text{SD}(Y) &= 4.10 \\
   \end{aligned}
   $$

   The only thing new to compute is $\overline{XY}$.
   $$
   XY = [-2,8,48], \qquad \overline{XY}=(-2+8+48)/3=18
   $$
   Therefore,$ r = \frac{18 - 3.66 \times 2.66}{2.05 \times 4.10} \approx 0.97$

   \end{block}
   \end{frame}

   %%%%%%%%%%%%%%%%%%%%%%%%%%%%%%%%%%%%%%%%%%%%%%%%%%%%%%%%%%%%

   \begin{frame} \frametitle{Correlation}

   \begin{block}
   {Properties}

   \begin{itemize}
   \item Correlation is unitless.

   \item Changing units of $X$ or $Y$ does not change the correlation.

   \item Correlation does not change if we interchange $X$ and $Y$.

   \end{itemize}

   \end{block}
   \end{frame}

   %CODE
       % from matplotlib import rc
   % import pylab, numpy as np, sys
   % np.random.seed(0);import random; random.seed(0)
   % import matplotlib.mlab as ML
   % H = ML.csv2rec('%s/pearson_lee.csv' % datadir)
   % M = H['mother']
   % D = H['daughter']
   % pylab.scatter(M, D, c='red')
   % pylab.gca().set_xlabel("Mother's height (inches)")
   % pylab.gca().set_ylabel("Daughter's height (inches)")
   % pylab.title("r=%0.2f" % np.corrcoef([D,M])[0,1])
   % 


   \begin{frame}
   \frametitle{Correlation}
   \begin{center}
   \resizebox{!}{2.7in}{\includegraphics{./images/inline/5b695bd691.pdf}}    
   \end{center}
   Correlation is symmetric
   \end{frame}

   %CODE
       % from matplotlib import rc
   % import pylab, numpy as np, sys
   % np.random.seed(0);import random; random.seed(0)
   % import matplotlib.mlab as ML
   % H = ML.csv2rec('%s/pearson_lee.csv' % datadir)
   % M = H['mother']
   % D = H['daughter']
   % pylab.scatter(D, M, c='red')
   % pylab.gca().set_ylabel("Mother's height (inches)")
   % pylab.gca().set_xlabel("Daughter's height (inches)")
   % pylab.title("r=%0.2f" % np.corrcoef([D,M])[0,1])
   % 


   \begin{frame}
   \frametitle{Correlation}
   \begin{center}
   \resizebox{!}{2.7in}{\includegraphics{./images/inline/e7bd5169f6.pdf}}    
   \end{center}
   Correlation is symmetric
   \end{frame}

   %CODE
       % from matplotlib import rc
   % import pylab, numpy as np, sys
   % np.random.seed(0);import random; random.seed(0)
   % import matplotlib.mlab as ML
   % H = ML.csv2rec('%s/pearson_lee.csv' % datadir)
   % M = H['mother']
   % D = H['daughter']
   % pylab.scatter(D, M, c='red')
   % pylab.gca().set_ylabel("Mother's height (inches)")
   % pylab.gca().set_xlabel("Daughter's height (inches)")
   % pylab.title("r=%0.2f" % np.corrcoef([D,M])[0,1])
   % 


   \begin{frame}
   \frametitle{Correlation}
   \begin{center}
   \resizebox{!}{2.7in}{\includegraphics{./images/inline/e7bd5169f6.pdf}}    
   \end{center}
   Correlation is not causation
   \end{frame}

   %CODE
       % from matplotlib import rc
   % import pylab, numpy as np, sys
   % np.random.seed(0);import random; random.seed(0)
   % X = np.random.standard_normal(30)
   % X.sort()
   % 
   % pylab.clf()
   % e = np.random.standard_normal(30)
   % Y = 2 + 2.5 * X + 0.5 * e
   % Z = Y * 1.
   % Z[-1] = -3
   % pylab.scatter(X[-1],Z[-1], s=55, c='r')
   % pylab.scatter(X[:-1],Y[:-1], s=40)
   % pylab.gca().set_title('r=%0.2f with outlier, %0.2f without' % (np.corrcoef(X,Z)[0,1], np.corrcoef(X[:-1],Y[:-1])[0,1]))
   % pylab.gca().set_xlim([1.1*X.min(),1.1*X.max()])
   % 


   \begin{frame}
   \frametitle{Correlation}
   \begin{center}
   \resizebox{!}{2.7in}{\includegraphics{./images/inline/c9834b0fde.pdf}}    
   \end{center}
   Outliers affect correlation
   \end{frame}

   %CODE
       % from matplotlib import rc
   % import pylab, numpy as np, sys
   % np.random.seed(0);import random; random.seed(0)
   % X = np.linspace(-2,2,50)
   % X += 0.1 * np.random.uniform(0,0.1,(50,)) - 0.05
   % X.sort()
   % 
   % pylab.clf()
   % e = np.random.standard_normal(50)
   % Y = - 1.5 * X**2 + 0.5 * e
   % pylab.scatter(X,Y, c='r', s=40)
   % pylab.gca().set_title('r=%0.2f' % np.corrcoef(X,Y)[0,1])
   % 


   \begin{frame}
   \frametitle{Correlation}
   \begin{center}
   \resizebox{!}{2.7in}{\includegraphics{./images/inline/8ac43bd9e5.pdf}}    
   \end{center}
   Correlation is a linear measure of association
   \end{frame}

   %%%%%%%%%%%%%%%%%%%%%%%%%%%%%%%%%%%%%%%%%%%%%%%%%%%%%%%%%%%%

   \begin{frame} \frametitle{Correlation}

   \begin{block}
   {The SD line}
   \begin{itemize}

   \item The SD line passes through the point of averages $(\bar{X}, \bar{Y})$
   and has
   $$
   \text{slope(SD line)} = \frac{\text{SD}(Y)}{\text{SD}(X)} \times \text{sign}(r(X,Y))
   $$


   \item For every one standardized unit increase of $X$, the SD line
   changes by one standardized unit of $Y$.  The direction of change is positive
   if $X$ and $Y$ are positively correlated, and negative
   if they are negatively correlated.

   \end{itemize}
   \end{block}
   \end{frame}

   %CODE
       % from matplotlib import rc
   % import pylab, numpy as np, sys
   % np.random.seed(0);import random; random.seed(0)
   % import matplotlib.mlab as ML
   % H = ML.csv2rec('%s/pearson_lee.csv' % datadir)
   % M = H['mother']
   % D = H['daughter']
   % pylab.scatter(M, D, c='red')
   % pylab.gca().set_xlabel("Mother's height (inches)")
   % pylab.gca().set_ylabel("Daughter's height (inches)")
   % Dbar = D.mean(); Dsd = np.sqrt(((D - Dbar)**2).mean())
   % Mbar = M.mean(); Msd = np.sqrt(((M - Mbar)**2).mean())
   % pylab.plot([Mbar-3.5*Msd,Mbar,Mbar+3.5*Msd],
   %            [Dbar-3.5*Dsd,Dbar,Dbar+3.5*Dsd], 'b--', linewidth=3, label='SD line')
   % pylab.legend(['SD line'])
   % 


   \begin{frame}
   \frametitle{Correlation}
   \begin{center}
   \resizebox{!}{2.7in}{\includegraphics{./images/inline/848dba2167.pdf}}    
   \end{center}
   The cloud seems to cluster around the SD line
   \end{frame}

   %%%%%%%%%%%%%%%%%%%%%%%%%%%%%%%%%%%%%%%%%%%%%%%%%%%%%%%%%%%%

   \begin{frame} \frametitle{Correlation}

   \begin{block}
   {Ecological correlations}
   \begin{itemize}
   \item Plots of averages vs. averages can exaggerate correlations.
   \item This is because this is a plot of averages of X versus averages of Y, so the points are less variable.
   \item Example: in the next Figure, I divided heights into groups based on mother's height, then averaged both mother's height and daughter's height within that group.
   \end{itemize}
   \end{block}
   \end{frame}

   %CODE
       % from matplotlib import rc
   % import pylab, numpy as np, sys
   % np.random.seed(0);import random; random.seed(0)
   % import numpy as np, pylab
   % import matplotlib.mlab as ML
   % H = ML.csv2rec('%s/pearson_lee.csv' % datadir)
   % M = H['mother']
   % D = H['daughter']
   % 
   % Mm = []
   % Dm = []
   % for ml in range(56,68):
   %     g = (M >= ml-0.5) * (M < ml+0.5)
   %     Mm.append(M[g].mean())
   %     Dm.append(D[g].mean())
   % 
   % 
   % pylab.scatter(Mm, Dm, s=60, c='red')
   % pylab.gca().set_xlabel("Average mother's height within group (inches)")
   % pylab.gca().set_ylabel("Average daughter's height within group (inches)")
   % pylab.gca().set_title('r=%0.2f' % np.corrcoef(Mm,Dm)[0,1])
   % 
   % g = (M >= 64.5) * (M < 65.5)
   % pylab.scatter(M[g], D[g], c='blue')
   % xf, yf = pylab.poly_between([64.5,65.5], [50,50], [75, 75])
   % pylab.fill(xf, yf, facecolor='yellow', alpha=0.4, hatch='/')
   % 


   \begin{frame}
   \frametitle{Correlation}
   \begin{center}
   \resizebox{!}{2.7in}{\includegraphics{./images/inline/edcea38445.pdf}}    
   \end{center}
   Ecological correlations ignore variability \dots
   \end{frame}

   %CODE
       % from matplotlib import rc
   % import pylab, numpy as np, sys
   % np.random.seed(0);import random; random.seed(0)
   % import matplotlib.mlab as ML
   % D = ML.csv2rec('%s/independent_example.csv' % datadir)
   % X = D['x']
   % Y = D['y']
   % 
   % pylab.scatter(X,Y, c='r')
   % 


   \begin{frame}
   \frametitle{Correlation}
   \begin{center}
   \resizebox{!}{2.7in}{\includegraphics{./images/inline/f78353fd54.pdf}}    
   \end{center}
   Uncorrelated $X$ and $Y$
   \end{frame}

   %CODE
       % from matplotlib import rc
   % import pylab, numpy as np, sys
   % np.random.seed(0);import random; random.seed(0)
   % import matplotlib.mlab as ML
   % D = ML.csv2rec('%s/independent_example.csv' % datadir)
   % X = D['x']
   % Y = D['y']
   % xf, yf = pylab.poly_between([-4,4], [3.8,-4.2], [4.2, -3.8])
   % g = (X + Y >= -0.2) * (X + Y <= 0.2)
   % pylab.fill(xf, yf, facecolor='yellow', alpha=0.4, hatch='/')
   % 
   % pylab.scatter(X[g],Y[g], c='black')
   % pylab.scatter(X[~g],Y[~g], c='r')
   % pylab.gca().set_xlim([-4,4])
   % pylab.gca().set_ylim([-4,4])
   % 


   \begin{frame}
   \frametitle{Correlation}
   \begin{center}
   \resizebox{!}{2.7in}{\includegraphics{./images/inline/c89b576039.pdf}}    
   \end{center}
   Fixing the sum selects this diagonal strip
   \end{frame}

   %CODE
       % from matplotlib import rc
   % import pylab, numpy as np, sys
   % np.random.seed(0);import random; random.seed(0)
   % import matplotlib.mlab as ML
   % D = ML.csv2rec('%s/independent_example.csv' % datadir)
   % X = D['x']
   % Y = D['y']
   % g = (X + Y >= -0.2) * (X + Y <= 0.2)
   % pylab.scatter(X[g], Y[g], c='black')
   % pylab.title("r=%0.2f" % np.corrcoef([X[g],Y[g]])[0,1])
   % pylab.gca().set_xlim([-4,4])
   % pylab.gca().set_ylim([-4,4])
   % 


   \begin{frame}
   \frametitle{Correlation}
   \begin{center}
   \resizebox{!}{2.7in}{\includegraphics{./images/inline/488445d5e7.pdf}}    
   \end{center}
   Fixing the sum selects this diagonal strip
   \end{frame}

   %%%%%%%%%%%%%%%%%%%%%%%%%%%%%%%%%%%%%%%%%%%%%%%%%%%%%%%%%%%%

   \begin{frame} 

   \end{frame}

   \end{document}
