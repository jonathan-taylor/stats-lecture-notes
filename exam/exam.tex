%% This file was auto-generated by IPython.
%% Conversion from the original notebook file:
%% Final exam.ipynb
%%
\documentclass[11pt,english]{article}

%% This is the automatic preamble used by IPython.  Note that it does *not*
%% include a documentclass declaration. The documentclass is added at runtime
%% to the overall document.

\usepackage{amsmath}
\usepackage{amssymb}
\usepackage{graphicx}
\usepackage{ucs}
\usepackage[utf8x]{inputenc}

% needed for markdown enumerations to work
\usepackage{enumerate}

% Slightly bigger margins than the latex defaults
\usepackage{geometry}
\geometry{verbose,tmargin=3cm,bmargin=3cm,lmargin=2.5cm,rmargin=2.5cm}

% Define a few colors for use in code, links and cell shading
\usepackage{color}
\definecolor{orange}{cmyk}{0,0.4,0.8,0.2}
\definecolor{darkorange}{rgb}{.71,0.21,0.01}
\definecolor{darkgreen}{rgb}{.12,.54,.11}
\definecolor{myteal}{rgb}{.26, .44, .56}
\definecolor{gray}{gray}{0.45}
\definecolor{lightgray}{gray}{.95}
\definecolor{mediumgray}{gray}{.8}
\definecolor{inputbackground}{rgb}{.95, .95, .85}
\definecolor{outputbackground}{rgb}{.95, .95, .95}
\definecolor{traceback}{rgb}{1, .95, .95}

% Framed environments for code cells (inputs, outputs, errors, ...).  The
% various uses of \unskip (or not) at the end were fine-tuned by hand, so don't
% randomly change them unless you're sure of the effect it will have.
\usepackage{framed}

% remove extraneous vertical space in boxes
\setlength\fboxsep{0pt}

% codecell is the whole input+output set of blocks that a Code cell can
% generate.

% TODO: unfortunately, it seems that using a framed codecell environment breaks
% the ability of the frames inside of it to be broken across pages.  This
% causes at least the problem of having lots of empty space at the bottom of
% pages as new frames are moved to the next page, and if a single frame is too
% long to fit on a page, it will completely stop latex from compiling the
% document.  So unless we figure out a solution to this, we'll have to instead
% leave the codecell env. as empty.  I'm keeping the original codecell
% definition here (a thin vertical bar) for reference, in case we find a
% solution to the page break issue.

%% \newenvironment{codecell}{%
%%     \def\FrameCommand{\color{mediumgray} \vrule width 1pt \hspace{5pt}}%
%%    \MakeFramed{\vspace{-0.5em}}}
%%  {\unskip\endMakeFramed}

% For now, make this a no-op...
\newenvironment{codecell}{}

 \newenvironment{codeinput}{%
   \def\FrameCommand{\colorbox{inputbackground}}%
   \MakeFramed{\advance\hsize-\width \FrameRestore}}
 {\unskip\endMakeFramed}

\newenvironment{codeoutput}{%
   \def\FrameCommand{\colorbox{outputbackground}}%
   \vspace{-1.4em}
   \MakeFramed{\advance\hsize-\width \FrameRestore}}
 {\unskip\medskip\endMakeFramed}

\newenvironment{traceback}{%
   \def\FrameCommand{\colorbox{traceback}}%
   \MakeFramed{\advance\hsize-\width \FrameRestore}}
 {\endMakeFramed}

% Use and configure listings package for nicely formatted code
\usepackage{listingsutf8}
\lstset{
  language=python,
  inputencoding=utf8x,
  extendedchars=\true,
  aboveskip=\smallskipamount,
  belowskip=\smallskipamount,
  xleftmargin=2mm,
  breaklines=true,
  basicstyle=\small \ttfamily,
  showstringspaces=false,
  keywordstyle=\color{blue}\bfseries,
  commentstyle=\color{myteal},
  stringstyle=\color{darkgreen},
  identifierstyle=\color{darkorange},
  columns=fullflexible,  % tighter character kerning, like verb
}

% The hyperref package gives us a pdf with properly built
% internal navigation ('pdf bookmarks' for the table of contents,
% internal cross-reference links, web links for URLs, etc.)
\usepackage{hyperref}
\hypersetup{
  breaklinks=true,  % so long urls are correctly broken across lines
  colorlinks=true,
  urlcolor=blue,
  linkcolor=darkorange,
  citecolor=darkgreen,
  }

% hardcode size of all verbatim environments to be a bit smaller
\makeatletter
\g@addto@macro\@verbatim\small\topsep=0.5em\partopsep=0pt
\makeatother

% Prevent overflowing lines due to urls and other hard-to-break entities.
\sloppy


%% Adding user preamble from file:
%% /Users/jonathantaylor/Desktop/stats306b/latex/header.tex

\newcommand{\F}{{\cal F}}
\newcommand{\D}{{\cal D}}
\newcommand{\CGF}{\Lambda}
\newcommand{\norm}[1]{\lVert #1 \rVert}
\newcommand{\innerp}[2]{\langle #1 , #2 \rangle}
\newcommand{\argmax}{\mathop{\mathrm{argmax}}}
\newcommand{\argmin}{\mathop{\mathrm{argmin}}}
\newcommand{\minimize}{\mathop{\mathrm{minimize}}}
\newcommand{\maximize}{\mathop{\mathrm{maximize}}}  
\newcommand{\Ee}{\mathbb{E}}
\newcommand{\Qq}{\mathbb{Q}}
\newcommand{\Zz}{\mathbb{Z}}
\newcommand{\Pp}{\mathbb{P}}
\newcommand{\real}{\mathbb{R}}
\newcommand{\Vv}{\text{Var}}
\newcommand{\Mm}{{\cal M}}
\newcommand{\I}{I}

% From Rudy's scribed notes



%%%%%%%%%%%%%%%%%%%%%%%%%%
% SET MATH OPERATIONS
%\def\aa{\boldsymbol{a}}
\def\aa{{a}}
%\def\AA{\boldsymbol{A}}
\def\AA{{A}}
%\def\alphabold{\boldsymbol{\alpha}}
\def\alphabold{{\alpha}}
\def\argmax{\mathop{\rm argmax}\nolimits}
%\def\bb{\boldsymbol{b}}
\def\bb{{b}}
%\def\betabold{\boldsymbol{\beta}}
\def\betabold{{\beta}}
\def\betahat{\hat{\beta}}
%\def\betahatbold{\hat{\boldsymbol \beta}}
\def\betahatbold{\hat{\beta}}
\def\cov{{\mathop{\bf cov}\nolimits}}
\def\Cov{{\mathop{\bf Cov}\nolimits}}
\def\diag{\mathop{\rm diag}\nolimits}
%\def\DD{\boldsymbol{D}}
\def\DD{{D}}
%\def\ee{\boldsymbol{e}}
\def\ee{{e}}
\def\EE{\Ee}
%\def\etabold{\boldsymbol{\eta}}
\def\etabold{\eta}
\def\FFF{\mathcal{F}}
%\def\futilde{\underset{\widetilde{}}{\boldsymbol{f}}}
\def\futilde{\underset{\widetilde{}}{{f}}}
%\def\Gammabold{\boldsymbol{\Gamma}}
\def\Gammabold{{\Gamma}}
%\def\hh{\boldsymbol{h}}
\def\hh{{h}}
%\def\HH{\boldsymbol{H}}
\def\HH{{H}}
%\def\ii{\boldsymbol{i}}
\def\ii{{i}}
\def\iid{\overset{\rm i.i.d.}{\sim}}
\def\kurtosis{{\mathop{\rm kurtosis}\nolimits}}
\def\Kurtosis{{\mathop{\rm Kurtosis}\nolimits}}
%\def\lambdabold{\boldsymbol{\lambda}}
\def\lambdabold{{\lambda}}
\def\LLL{\mathcal{L}}
%\def\mubold{\boldsymbol{\mu}}
\def\mubold{{\mu}}
\def\mudot{\dot{\mu}}
\def\muhat{\hat{\mu}}
%\def\muutilde{\underset{\widetilde{}}{\boldsymbol{\mu}}}
\def\muutilde{\underset{\widetilde{}}{{\mu}}}
\def\NNN{\mathcal{N}}
%\def\nubold{\boldsymbol{\nu}}
\def\nubold{{\nu}}
%\def\onebold{\boldsymbol{1}}
\def\onebold{{1}}
%\def\oo{\boldsymbol{o}}
\def\oo{{o}}
%\def\pibold{\boldsymbol{\pi}}
\def\pibold{{\pi}}
%\def\piutilde{\underset{\widetilde{}}{\boldsymbol{\pi}}}
\def\piutilde{\underset{\widetilde{}}{{\pi}}}
%\def\pp{\boldsymbol{p}}
\def\pp{{p}}
\def\Prob{{\mathop{\rm Prob}\nolimits}}
\def\psidot{\dot{\psi}}
\def\psidoubledot{\overset{..}{\psi}}
\def\sd{\mathop{\rm sd}\nolimits}
\def\se{\mathop{\rm se}\nolimits}
\def\sign{\mathop{\rm sign}\nolimits}
\def\skew{{\mathop{\rm skew}\nolimits}}
\def\Skew{{\mathop{\rm Skew}\nolimits}}
%\def\ss{\boldsymbol{s}}
\def\ss{{s}}
%\def\sutilde{\underset{\widetilde{}}{\boldsymbol{s}}}
\def\sutilde{\underset{\widetilde{}}{{s}}}
%\def\tbold{\boldsymbol{t}}
\def\tbold{{t}}
%\def\tboldarrow{\overset{\rightarrow}{\boldsymbol{t}}}
\def\tboldarrow{\overset{\rightarrow}{{t}}}
%\def\thetabold{\boldsymbol{\theta}}
\def\thetabold{\theta}
%\def\Thetabold{\boldsymbol{\Theta}}
\def\Thetabold{{\Theta}}
\def\thetahat{\hat{\theta}}
%\def\TT{\boldsymbol{T}}
\def\TT{{T}}
\def\var{{\mathop{\rm var}\nolimits}}
\def\Var{{\mathop{\rm Var}\nolimits}}
\def\Vdot{\dot{V}}
\def\Vdoubledot{\overset{..}{V}}
%\def\VV{\boldsymbol{V}}
\def\VV{{V}}
\def\Xtilde{\underset{\widetilde{}}{\xx}}
%\def\xx{\boldsymbol{x}}
\def\xx{{x}}
%\def\xxarrow{\overset{\rightarrow}{\boldsymbol{x}}}
\def\xxarrow{\overset{\rightarrow}{{x}}}
\def\XX{X}
%\def\XX{\boldsymbol{X}}
\def\ybar{\bar{y}}
\def\Ybar{\bar{Y}}
%\def\ybarbold{\boldsymbol{{\bar{y}}}}
\def\ybarbold{{{\bar{y}}}}
\def\ytilde{\underset{\widetilde{}}{\yy}}
%\def\yy{\boldsymbol{y}}
\def\yy{{y}}
%\def\YY{\boldsymbol{Y}}
\def\YY{{Y}}
\def\YYY{\mathcal{Y}}
\def\zerobold{\boldsymbol{0}}
\def\zerobold{{0}}
\def\Zplus{\mathbb{Z}_+}
%\def\zz{\boldsymbol{z}}
\def\zz{{z}}


\def\etatilde{\underset{\widetilde{}}{\etabold}}
%\def\vtilde{\underset{\widetilde{}}{\boldsymbol{v}}}
\def\vtilde{\underset{\widetilde{}}{{v}}}
\def\bigVtilde{\underset{\widetilde{}}{\VV}}
\def\bigXtilde{\underset{\widetilde{}}{\XX}}
%%%%%%%%%%%%%%%%%%%%%%%%%%

\date{June 10, 2013}

\begin{document}



\title{STATS306B: Methods for Applied Statistics \\
Final Exam \\
} 
\maketitle


\section{Question 1}

A survey was carried out senior high school students, asking them if
they had ever used cigarettes, alcohol or marijuana. The resulting data
is summarized below:

\begin{codecell}
\begin{codeinput}
\begin{lstlisting}
%%R

# alcohol use
A = c(F,F,F,F,T,T,T,T)

# cigarette use
C = c(F,T,F,T,F,T,F,T)

# marijuana use
M = c(F,F,T,T,F,F,T,T)

# counts
Y = c(279,43,2,3,456,538,44,911)

D = xtabs(Y ~ A+C+M)
print(D)
\end{lstlisting}
\end{codeinput}
\begin{codeoutput}
\begin{verbatim}
, , M = FALSE

       C
A       FALSE TRUE
  FALSE   279   43
  TRUE    456  538

, , M = TRUE

       C
A       FALSE TRUE
  FALSE     2    3
  TRUE     44  911
\end{verbatim}
\end{codeoutput}
\end{codecell}
\subsection{Part A}

Assume the sample consists of $N=2276$ independent draws from some
population. Each draw results in a triple $(A,C,M)$ of binary outcomes.
Write down an explicit exponential family model for this dataset,
including sufficient statistic, reference measure and cumulant
generating function.

\subsection{Part B}

Suppose we wish to test the hypothesis that, in this population, alcohol
use is independent of marijuana use given cigarette use. Write this
model as an exponential family as in A (i.e.~sufficient statistics,
reference measure, etc.). What is the relation between the two models?

\subsection{Part C}

Below is the result from \texttt{R}'s fitting a particular model to the
above count data. Describe how the quantities \texttt{Null deviance} and
\texttt{Residual deviance} are computed. You can assume that you have
been given the vector $\hat{\mu}$ of fitted means for this model.

\begin{codecell}
\begin{codeinput}
\begin{lstlisting}
%%R
model1 = glm(Y~M+A+C,family=poisson())
print(summary(model1))
\end{lstlisting}
\end{codeinput}
\begin{codeoutput}
\begin{verbatim}
Call:
glm(formula = Y ~ M + A + C, family = poisson())

Deviance Residuals: 
      1        2        3        4        5        6        7        8  
 19.639   -8.436   -8.832  -12.440    3.426   -7.817  -17.683   14.522  

Coefficients:
            Estimate Std. Error z value Pr(>|z|)    
(Intercept)  4.17254    0.06496  64.234  < 2e-16 ***
MTRUE       -0.31542    0.04244  -7.431 1.08e-13 ***
ATRUE        1.78511    0.05976  29.872  < 2e-16 ***
CTRUE        0.64931    0.04415  14.707  < 2e-16 ***
---
Signif. codes:  0 ‘***’ 0.001 ‘**’ 0.01 ‘*’ 0.05 ‘.’ 0.1 ‘ ’ 1 

(Dispersion parameter for poisson family taken to be 1)

    Null deviance: 2851.5  on 7  degrees of freedom
Residual deviance: 1286.0  on 4  degrees of freedom
AIC: 1343.1

Number of Fisher Scoring iterations: 6
\end{verbatim}
\end{codeoutput}
\end{codecell}
\subsection{Part D}

Below are 4 other models fit to the same data. Which of these models
\texttt{(model1, model2, model3, model4, model5)}, if any, correspond to
your model in Part A? Which, if any, correspond to the model in Part B?

\begin{codecell}
\begin{codeinput}
\begin{lstlisting}
%%R
# independence
model2 = glm(Y~A*(M+C), family=poisson())
model3 = glm(Y~C*(A+M), family=poisson())
model4 = glm(Y~C*(M+A), family=poisson())
model5 = glm(Y~C*M*A, family=poisson())


\end{lstlisting}
\end{codeinput}
\end{codecell}
\subsection{Part E}

Describe how you would test the hypothesis described in Part B, on the
basis of having the fitted values $(\hat{\mu}_1, \dots, \hat{\mu}_5)$ of
the above 5 models.

\section{Question 2}

In this question, we consider a $2$-component normal mixture model on
$\mathbb{R}$ which can be described by simulating from the following
distribution: \[
\begin{aligned}
Z | \pi & \sim \text{Bernoulli}(\pi)  \\
Y|Z=i,\mu_i,\sigma_i & \sim N(\mu_i, \sigma^2_i).
\end{aligned}
\]

In a frequentist model, we must estimate
$(\mu_1, \mu_2, \sigma^2_1, \sigma^2_2, \pi)$. In this question, we
consider a Bayesian model with a prior on $\pi$ and, for simplicity,
consider $(\mu_1,\mu_2, \sigma^2_1, \sigma^2_2)$ to be fixed. The prior
on $\pi$ is taken to be $\text{Beta}(\alpha,\beta)$.

\subsection{Part A}

Write out the log-likelihood based on observing an IID sample of size
$n$ from this model. That is, the observed data is
$Y=(Y_1, \dots, Y_n)$. What are the parameters in the likelihood? Does
the posterior distribution of $\pi|Y$ have a simple form?

\subsection{Part B}

Write out the joint distribution of $(Y,Z,\pi)$ where
$Z=(Z_1,\dots,Z_n)$ are the unobserved assignments in the hierarchical
model. Up to a normalizing constant, what is the conditional
distribution $(Z,\pi)|Y$?

\subsection{Part C}

Describe the Gibbs sampler for sampling from the above posterior
distribution. Instead of sampling a coordinate at a time, describe how
to sample all the $Z$'s having fixed $\pi$, then how to sample $\pi$
having fixed all the $Z$'s. How can this be used to estimate
\[\mathbb{E}(h(\pi)|Y)\] for some $h:\mathbb{R} \rightarrow \mathbb{R}$?
What parameters does the above expectation depend on?

\subsection{Part D}

Suppose we included priors on $(\mu_1, \mu_2, \sigma_1^2, \sigma_2^2)$.
Describe the main changes to the Gibbs sampler in Parts A-C.

\section{Question 3}

Suppose we observe measurements of the form \[
\begin{aligned}
Y_i &= \mu + \alpha_i + \epsilon_i, \qquad 1 \leq i \leq n\\
\end{aligned}
\] where $\mu \in \mathbb{R}$ is a \emph{fixed effect},
$\alpha \in \mathbb{R}^n$ are \emph{random effects} and
$\epsilon \in \mathbb{R}^n$ are \emph{measurement errors}.

We make the following assumptions \[
\begin{pmatrix}\alpha \\ \epsilon \end{pmatrix}  \sim N\left( 0, \begin{pmatrix}  \gamma^2 \cdot I_{n \times n}  & 0 \\ 0 & \text{diag}(\sigma^2_1, \dots, \sigma^2_n) \end{pmatrix}\right)
\] where the $\sigma^2_i$'s are assumed known from the measurement
device. The unknowns in the problem are $(\mu,\gamma^2)$.

\subsection{Part A}

Write out the log-likelihood, score and Fisher information for the
parameters $(\mu,\gamma^2)$.

\subsection{Part B}

Outline the steps of a Newton-Raphson algorithm to find the MLE of
$(\mu, \gamma^2)$.

\subsection{Part C}

Having estimated $(\mu,\gamma^2)$ describe how to construct approximate
95\% confidence intervals for each of $(\mu,\gamma^2)$.

\section{Question 4}

Consider the same random effects model as Question 3. In this question,
we will approach the problem using the EM algorithm.

\subsection{Part A}

Consider the random variables $\alpha$ as \emph{missing}. Write out the
complete likelihood for $(\mu,\gamma^2)$ as an exponential family. What
is the complete data in this setting? What are the sufficient
statistics?

\subsection{Part B}

What is the conditional distribution of $\alpha | Y$ where
$Y=(Y_1, \dots, Y_n)$ is the observed data? Use this information to
carry out the $E$ step. That is, given current values
$(\mu_{\text{cur}}, {\gamma^2}_{\text{cur}})$ work out the appropriate
conditional expectations of the sufficient statistics from Part A given
the observed data.

\subsection{Part C}

Describe the $M$-step of this EM algorithm. That is, given the
pseudo-sufficient statistics from Part B, how do we update our estimate
of $(\mu, \gamma^2)$?

\subsection{Part D}

Describe the full EM algorithm to estimate $(\mu,\gamma^2)$.

\subsection{Part E}

Having used the EM algorithm to estimate $(\mu, \gamma^2)$, describe how
to construct approximate 95\% confidence intervals for each of
$(\mu,\gamma^2)$.


\end{document}
